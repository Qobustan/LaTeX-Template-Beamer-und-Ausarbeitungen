\documentclass{beamer}

% Hier befinden sich alle Metadaten; Pakete, Einstellungen und die sonstigen Ressourcen
\input{header.tex}
\usepackage[utf8]{inputenc}

% ANLEITUNG: Passen Sie die Sprache an Ihre Bedürfnisse an (z.B. english statt ngerman)
\usepackage[ngerman]{babel}
\usepackage[backend=bibtex, style=alphabetic]{biblatex}
\addbibresource{Vortrag.bib}

% ANLEITUNG: Wählen Sie ein Beamer-Theme (Madrid, Berlin, Copenhagen, etc.)
\usetheme{Madrid}

% Den \pause - Befehl steuern
% Vortragsversion -> true; Druckversion -> false
\newif\ifpauseswitch
\pauseswitchtrue % setze auf \pause aktivieren

\begin{document}
	
	\maketitle
	
	% TikZ - Bibliothek
	\usetikzlibrary{arrows.meta}
	\usetikzlibrary{shapes,arrows.meta,positioning}
	
	% Zeige Inhaltsverzeichnis zu Beginn jeder Section
	\AtBeginSection[] % Do nothing for \section*
	{
		\begin{frame}<beamer>
			\frametitle{Outline}
			\tableofcontents[currentsection]
		\end{frame}
	}

% ============================================================
% BEISPIELINHALT: Nichtparametrische Dichteschätzung
% ANLEITUNG: Ersetzen Sie den folgenden Inhalt mit Ihrem eigenen Vortrag.
%            Dieser Inhalt dient als Beispiel für die Verwendung der Vorlage.
% ============================================================

\section{Einleitung \& Motivation}

\begin{frame}{Parametrische vs. Nichtparametrische Dichteschätzung (\"Ubersicht)}
	\begin{columns}[T]
		% Parametrisch
		\begin{column}{0.45\textwidth}
			\textbf{Parametrisch (kennen wir schon $\rightarrow$ Stochastik)}
			\begin{itemize}
				\item Annahme: Verteilung ist bekannt!
				\item Wenige Parameter ($\mu, \sigma^2$)
				\item Schätzung durch Parametersch\"atzer wie bspw. Maximum-Likelihood
			\end{itemize}
		\end{column}
		
		\ifpauseswitch
			\pause
		\fi
		
		% Nichtparametrisch
		\begin{column}{0.45\textwidth}
			\textbf{Nichtparametrisch (KDE)}
			\begin{itemize}
				\item Keine feste Form; die Zufallsvariable $X$ muss aber stetig sein, die Dichte $f$ eine ''glatte'' Funktion
				\item Schätzung: Kern (wie bspw. Epanechnikov) sowie Bandbreite $h$
				\item Erfolgt mittels des Kerndichtesch\"atzers
			\end{itemize}
		\end{column}
	\end{columns}
	
	\vspace{1em}
	
	\centering
	\begin{tikzpicture}[>=stealth, thick]
		% Pfeile zwischen Konzepten
		\draw[<->] (-3,0) -- (3,0) node[midway,above]{\small Modellannahme vs. Datengetrieben};
		\draw[<->] (-3,-0.6) -- (3,-0.6) node[midway,above]{\small Einfachheit vs. Flexibilität};
	\end{tikzpicture}
	
\end{frame}

\begin{frame}
	\frametitle{Einleitung \& Motivation}
	
	\textbf{Warum nichtparametrisch? \cite{BüningTrenkler+1994}}
	\begin{itemize}
		\item Klassische parametrische Verfahren (z.\,B. unter Annahme einer Normalverteilung) sind oft zu starr.
		\begin{itemize}
			\item Starke Kopplung an unsicheren Annahmen!
		\end{itemize}
		\item Bei unbekannter Verteilung bietet die nichtparametrische Schätzung oft bessere Einblicke in die Datenstruktur.
	\end{itemize}
	
	\ifpauseswitch
		\pause
	\fi
	
	\textbf{Zielsetzung:}
	\begin{itemize}
		\item Annäherung an den Wert $f(x)$ an einer Stelle $x$, ohne eine spezifische parametrische Familie vorauszusetzen.
	\end{itemize}
\end{frame}

\section{Der Rosenblatt-Schätzer}

\begin{frame}
	\frametitle{Der Rosenblatt-Schätzer (Die theoretische Basis)}
	
	\begin{itemize}
		\item \textbf{Idee:} Da $f(x) = F'(x)$, nutzt man den Differenzenquotienten der empirischen Verteilungsfunktion $F_n$.
	\end{itemize}
\end{frame}

\begin{frame}
	\frametitle{Der Rosenblatt-Schätzer (Formel)}
	
	\begin{itemize}
		\item \textbf{Der Schätzer:}
		\begin{equation}
			f_n(x) = \frac{F_n(x+h) - F_n(x-h)}{2h}
		\end{equation}
	\end{itemize}
\end{frame}

\begin{frame}
	\frametitle{Der Rosenblatt-Schätzer (Konsistenz)}
	
	\begin{itemize}
		\item \textbf{Konsistenz:} Damit der Schätzer gegen die wahre Dichte konvergiert, muss für $n \to \infty$ gelten:
		\begin{itemize}
			\item Die Bandbreite $h \to 0$ (Bias reduzieren).
			\item $nh \to \infty$ (Varianz reduzieren).
		\end{itemize}
	\end{itemize}
\end{frame}

\section{Das Histogramm}

\begin{frame}
	\frametitle{Das Histogramm (Der Klassiker)}
	
	\begin{itemize}
		\item \textbf{Funktionsweise:} Zerlegung des Datenraums in Boxen (Bins) der Breite $h$. Die Höhe der Box entspricht der relativen Häufigkeit $\frac{n_k}{n}$ normiert durch $h$.
	\end{itemize}
\end{frame}

\begin{frame}
	\frametitle{Das Histogramm (Probleme)}
	\begin{itemize}
		\item \textbf{Probleme:}
		\begin{itemize} % statt enumerate
			\item Unstetigkeit: Die geschätzte Dichte ist eine Treppenfunktion.
			\item Subjektivität: Abhängig von $x_0$ und $h$.
		\end{itemize}
	\end{itemize}
\end{frame}


\begin{frame}
	\frametitle{Das Histogramm (Optimale Bandbreite)}
	
	\begin{itemize}
		\item \textbf{Optimale Bandbreite ($h$):}
		\begin{itemize}
			\item Scott: $h \approx 3.49 s n^{-1/3}$
			\item Freedman-Diaconis: $h^* = 2(x_{0.75} - x_{0.25})n^{-1/3}$
		\end{itemize}
	\end{itemize}
\end{frame}

\section{Kernschätzer (KDE)}

\begin{frame}
	\frametitle{Kernschätzer (KDE) - Einführung}
	
	\begin{itemize}
		\item \textbf{Der Schätzer:}
		\begin{equation}
			\hat{f}_n(x) = \frac{1}{nh} \sum_{i=1}^{n} K\left(\frac{x - X_i}{h}\right)
		\end{equation}
	\end{itemize}
\end{frame}

\begin{frame}
	\frametitle{Kernschätzer (KDE) - Wahl des Kerns}
	
	\begin{itemize}
		\item \textbf{Kernwahl:} Der Kern muss zu 1 integrieren ($\int K(x) dx = 1$).
		\item Gängige Kerne:
		\begin{itemize}
			\item Rechteck-Kern
			\item Gauß-Kern
			\item Epanechnikov-Kern
		\end{itemize}
	\end{itemize}
\end{frame}

\begin{frame}
	\frametitle{Kernschätzer (KDE) - Bandbreitenwahl}
	
	\begin{itemize}
		\item \textbf{Bandbreitenwahl (Silverman's Rule):}
		\begin{equation}
			h_{opt} \approx 1.06 \sigma n^{-1/5}
		\end{equation}
	\end{itemize}
\end{frame}

\section{Nichtparametrische Regression}

\begin{frame}
	\frametitle{Nichtparametrische Regression - Einführung}
	
	\textbf{Watson-Nadaraya Schätzer:}
	\begin{equation}
		\hat{m}_{WN}(x) = \frac{\sum_{i=1}^n K\left(\frac{x - X_i}{h}\right) Y_i}{\sum_{i=1}^n K\left(\frac{x - X_i}{h}\right)}
	\end{equation}
	\begin{itemize}
		\item Schätzung von $m(x)$ als gewichteter Mittelwert der umliegenden $Y$-Werte.
	\end{itemize}
\end{frame}

\section{Robuste Lineare Regression}

\begin{frame}
	\frametitle{Robuste Lineare Regression (Theil-Schätzer) - Einführung}
	
	\begin{itemize}
		\item \textbf{Theil-Methode I:} Steigung zwischen Paaren $(i, i+n/2)$, Median als Schätzer.
		\item \textbf{Theil-Sen-Schätzer (Methode II):} Median der Steigungen zwischen allen Paaren $i < j$:
	\end{itemize}
\end{frame}

\begin{frame}
	\frametitle{Robuste Lineare Regression (Theil-Schätzer) - Formel}
	
	\begin{equation}
		H_{ij} = \frac{Y_j - Y_i}{X_j - X_i}
	\end{equation}
	\begin{itemize}
		\item Der Schätzer für den Anstieg $\beta$ ist der Median dieser Steigungen.
	\end{itemize}
\end{frame}

\section{Zusammenfassung}

\begin{frame}
	\frametitle{Fazit und Ausblick }
	
	\begin{itemize}
		\item \textbf{Flexibilität:} Nichtparametrische Methoden passen sich den Daten an.
		\item \textbf{Parameter $h$:} Kritischer Schritt, Balance zwischen Bias und Varianz.
	\end{itemize}
\end{frame}

\begin{frame}
	\frametitle{Fazit und Ausblick (Fortsetzung)}
	
	\begin{itemize}
		\item \textbf{Robustheit:} Verfahren wie der Theil-Sen-Schätzer bieten robuste Alternativen zur klassischen Regression, besonders bei Ausreißern.
	\end{itemize}
\end{frame}

\section{Literaturverzeichnis}
	
	\begin{frame}[allowframebreaks]{Literaturverzeichnis}
		% Referenzen
		\printbibliography
	\end{frame}

% ============================================================
% ENDE DES BEISPIELINHALTS
% ============================================================
	
\end{document}
